
\documentclass{bioinfo}
\copyrightyear{2005}
\pubyear{2005}

\usepackage{listings}
\usepackage{color}
\definecolor{lightgray}{rgb}{.9,.9,.9}
\definecolor{darkgray}{rgb}{.4,.4,.4}
\definecolor{purple}{rgb}{0.65, 0.12, 0.82}
\lstdefinelanguage{JavaScript}{
  keywords={break, case, catch, continue, debugger, default, delete, do, else, false, finally, for, function, if, in, instanceof, new, null, return, switch, this, throw, true, try, typeof, var, void, while, with},
  morecomment=[l]{//},
  morecomment=[s]{/*}{*/},
  morestring=[b]',
  morestring=[b]",
  ndkeywords={class, export, boolean, throw, implements, import, this},
  keywordstyle=\color{blue}\bfseries,
  ndkeywordstyle=\color{darkgray}\bfseries,
  identifierstyle=\color{black},
  commentstyle=\color{purple}\ttfamily,
  stringstyle=\color{red}\ttfamily,
  sensitive=true
}

\lstset{
   language=JavaScript,
   backgroundcolor=\color{lightgray},
   extendedchars=true,
   basicstyle=\footnotesize\ttfamily,
   showstringspaces=false,
   showspaces=false,
   numbers=left,
   numberstyle=\footnotesize,
   numbersep=9pt,
   tabsize=2,
   breaklines=true,
   showtabs=false,
   captionpos=b
}
\usepackage[ngerman]{babel}
%\usepackage[utf8]{inputenc}
\begin{document}
\raggedright
\firstpage{1}

\title[short Title]{Javascript Technology: Module Pattern}
\author[Sample \textit{et~al}]{Corresponding Author\,$^{1,*}$, Co-Author\,$^{2}$ and Co-Author\,$^2$\footnote{to whom correspondence should be addressed}}
\address{$^{1}$Department of XXXXXXX, Address XXXX etc.\\
$^{2}$Department of XXXXXXXX, Address XXXX etc.}

\history{Received on XXXXX; revised on XXXXX; accepted on XXXXX}

\editor{Associate Editor: XXXXXXX}

\maketitle

\begin{abstract}

\section{Motivation:}
This paper provides an in depth description of the Javascript Module Pattern. The Module Pattern is a way to implement classlike behaviour in Javascript by using the closure of anonymous functions to provide privacy and scope. It also explains the advantages and disadvantages that come with using the Module Pattern for your application. Finally it presents an application that has used the Module Pattern.

\section{Contact:} \href{name@bio.com}{name@bio.com}
\end{abstract}

\section{Introduction}
Since Javascript was not designed with object orientation in mind certain tricks are required. The Module Pattern is one such trick. It implements classlike behaviour by allowing public and private properties in one single datastructure. This keeps the properties out of the global namespace and prevents unwanted modification from outside the structure. The Module Pattern is an entirely separate alternative to Javascript objects.


\section{Module Pattern:}


\subsection{Overview:}
As mentioned in the Introduction, the Module Pattern allows public and private properties in one datastructure. This is achieved by creating an object inside an anonymous function, called "module" in the code example, that is executed immediately after it's definition. The object is returned at the end of the function. Everything in this function exists inside a closure, providing privacy and state.  \\

\begin{flushleft}
All public methods and variables are defined as part of the object, all private ones are created independent of it. The return value is saved in a variable thus allowing it's properties to be accessed from outside the function's scope. The state provided by the closure remains consistent over all invocations of the Module. The following codes illustrates the basic Module Pattern.
\end{flushleft}

\begin{lstlisting}
//define anonymous function
var My_Module = (function(){
	var module = {};
	var private_variable = 3;
	module.public_variable = 9;
	
	//return public part of the module
	return module;
	
//execute anonymous function immediately
]());
\end{lstlisting}

Public properties can now be accessed like this:

\begin{lstlisting}
My_Module.public_property
\end{lstlisting}


While private properties can not be accessed from outside the anonymous function's closure, the following code will not produce an error, but instead create a new public variable named "private\_variable"
\begin{lstlisting}
My_Module.private_variable
\end{lstlisting}

\subsection{Strengths and Weaknesses of the Module Pattern:}
A big advantage of the Module Pattern is scalabilty. Modules are isolated pieces of code and can be added or removed fairly easily since they are mostly independent of other code. The isolation also allows for a simple distribution of work among several programmers as they can be assigned different Modules to implement and can work separately.\\

\begin{flushleft}
Restricting variables to a local scope leads to less clutter in the global namespace. This, in addition to the fact that the public variables are bound to one module variable, prevents variable name conflicts which can be a problem when importing libraries or working with a team of developers. On top of that Modules can be augmented to add more methods and variables when required.
\end{flushleft}

The Module Pattern also has a few downsides. For one, inheritance requires the inheriting Module to explicitly copy all properties of the super Module. Also it is not possible to manipulate the private properties of a Module from outside the Module's scope. Not even while augmenting it.\\ 

Another problem is that changing the visibility of a property requires the programmer to edit every line of code that contains said property. This is the case because visibility is not defined by a single keyword but by whether the property is part of the returned object inside the Module so it is either accessed by "module.property" if it is public or just "property" if it is private.

%add code examples for strengths/weaknesses

\subsection{Global Variables:}
Global variables can make code hard to read or maintain since it is difficult for humans to determine where in the code they are used. The Module Pattern allows to import global variables into a Module explicitly by using them as arguments for the Module's anonymous function. Of course global variables can be accessed inside the function's closure regardless of whether they are imported explicitly or not. Using global variables as arguments is done for increased readability and is highly recommended\\

\begin{lstlisting}
//use global variable as function parameter
var My_Module = (function(global_variable){
	var module = {};
	
	//access global variable
	module.global_incremented = function(){
		return global_variable + 1;
	}
	
	return module;	
	
//use global variable as argument
](global_variable));
\end{lstlisting}

\subsection{Inheritance:}
Modules inherit from other Modules by copying all their non-private properties. To achieve this the super Module's public part is imported into the inheriting Module as an argument. A reference to the super Module is saved and all it's properties are recreated.

\begin{lstlisting}
//define anonymous function with
//super module as parameter
var My_Module = (function(super_module){
	var module = {};
	
	//create reference to super module
	module.super = super_module;
	
	//copy all properties from the 
	//super module to the new module
	for (key in super_module) {
		if (super_module.hasOwnProperty(key) {
			module[key] = super_module[key];
		}
	}
	
//import super module as argument
//to new anonymous function
](super_module));
\end{lstlisting}


\subsection{Augmentation:}
As was briefly touched upon in the "Strengths and Weaknesses of the Module Pattern" section, Modules can be augmented after their original definition. A new anonymous function is defined and executed immediately. It takes the variable that held the original Module as an argument. Instead of creating a new object to contain all public properties, new public properties are added to the original modules object. The public properties of the original Module can also be modified. Said object is returned at the end of the function and the return value is used to overwrite the variable that held the original Module's public properties. The new public properties can now be accessed along the original ones through this variable. \\



\begin{lstlisting}
//define anonymous function with
//original module as parameter
var Original_Module = (function(original_module){
	
	//add new properties	
	original_module.new_variable = 25;
	original_module.new_method = function(){
		return 3+5;
	};
	
	//overwrite original properties
	original_module.original_variable = 18;
	original_module.original_method = function(){
		return 4+2;
	}
	
	return original_module;
	
//import original module as argument
//to new anonymous function
](Original_Module));
\end{lstlisting}



It is worth noting that the original private properties can not be accessed in the new anonymous function, since they are only accessible from inside the original anonymous function's closure. New anonymous properties can be defined but they only exist in their own function's closure and can not be accessed from anywhere else.\\

These augmentations can be even be done from different files and can even be used when the original Module has not been created. This will be explained in detail in the following section.





\subsection{loading order: cross-file private state...}

\section{Our Web App(more descriptive title required - Overview:}

This is better looking sample text compared to the original sample text that just repeated the word "Text". This is better looking sample text compared to the original sample text that just repeated the word "Text". This is better looking sample text compared to the original sample text that just repeated the word "Text".


\section{How Module Pattern Helped Our App (better title pls, jeebus)}

This is better looking sample text compared to the original sample text that just repeated the word "Text". This is better looking sample text compared to the original sample text that just repeated the word "Text". This is better looking sample text compared to the original sample text that just repeated the word "Text".


\section{Conclusion/Summary}

This is better looking sample text compared to the original sample text that just repeated the word "Text". This is better looking sample text compared to the original sample text that just repeated the word "Text". This is better looking sample text compared to the original sample text that just repeated the word "Text".
%Einleuitung / Was ist ModulePattern / Was ist daran gut, was schlecht / vll paar Beispiele wie man das elegant anwendet / beschreibung unserer WebApp und warum das pattern da geholfen hat / Schluss

\section{everything past this line is a relict from the default template}

\section{Approach}

Equation~(\ref{eq:01}) Text Text Text Text Text Text  Text Text Text Text Text Text Text Text Text  Text Text Text Text Text Text. Figure \ref{fig:02} shows that the above method  Text Text Text Text  Text Text Text Text Text Text  Text Text.  \citealp{Boffelli03} might want to know about  text text text text ��


\begin{methods}
\section{Methods}

Text Text Text Text Text Text  Text Text Text Text Text Text Text Text Text  Text Text Text Text Text Text. Figure \ref{fig:02} shows that the above method  Text Text Text Text  Text Text Text Text Text Text  Text Text.  \citealp{Boffelli03} might want to know about  text text text text
Text Text Text Text Text Text  Text Text Text Text Text Text Text Text Text  Text Text Text Text Text Text. Figure \ref{fig:02} shows that the above method  Text Text Text Text  Text Text Text Text Text Text  Text Text.  \citealp{Boffelli03} might want to know about  text text text text
Text Text Text Text Text Text  Text Text Text Text Text Text Text Text Text  Text Text Text Text Text Text. Figure \ref{fig:02} shows that the above method  Text Text Text Text  Text Text Text Text Text Text  Text Text.  \citealp{Boffelli03} might want to know about  text text text text

\begin{itemize}
\item for bulleted list, use itemize
\item for bulleted list, use itemize
\item for bulleted list, use itemize
\end{itemize}



Text Text Text Text Text Text  Text Text Text Text Text Text Text Text Text  Text Text Text Text Text Text. Figure \ref{fig:02} shows that the above method  Text Text Text Text  Text Text Text Text Text Text  Text Text.  \citealp{Boffelli03} might want to know about  text text text text
Text Text Text Text Text Text  Text Text Text Text Text Text Text Text Text  Text Text Text Text Text Text. Figure \ref{fig:02} shows that the above method  Text Text Text Text  Text Text Text Text Text Text  Text Text.  \citealp{Boffelli03} might want to know about  text text text text
Text Text Text Text Text Text  Text Text Text Text Text Text Text Text Text  Text Text Text Text Text Text. Figure \ref{fig:02} shows that the above method  Text Text Text Text  Text Text Text Text Text Text  Text Text.  \citealp{Boffelli03} might want to know about  text text text text
Text Text Text Text Text Text  Text Text Text Text Text Text Text Text Text  Text Text Text Text Text Text. Figure \ref{fig:02} shows that the above method  Text Text Text Text  Text Text Text Text Text Text  Text Text.  \citealp{Boffelli03} might want to know about  text text text text
Text Text Text Text Text Text  Text Text Text Text Text Text Text Text Text  Text Text Text Text Text Text.


Text Text Text Text Text Text  Text Text Text Text Text Text Text Text Text  Text Text Text Text Text Text. Figure \ref{fig:02} shows that the above method  Text Text Text Text  Text Text Text Text Text Text  Text Text.  \citealp{Boffelli03} might want to know about  text text text text
Text Text Text Text Text Text  Text Text Text Text Text Text Text Text Text  Text Text Text Text Text Text. Figure \ref{fig:02} shows that the above method  Text Text Text Text  Text Text Text Text Text Text  Text Text.  \citealp{Boffelli03} might want to know about  text text text text
Text Text Text Text Text Text  Text Text Text Text Text Text Text Text Text  Text Text Text Text Text Text. Figure \ref{fig:02} shows that the above method  Text Text Text Text  Text Text Text Text Text Text  Text Text.  \citealp{Boffelli03} might want to know about  text text text text



Text Text Text Text Text Text  Text Text Text Text Text Text Text Text Text  Text Text Text Text Text Text. Figure \ref{fig:02} shows that the above method  Text Text Text Text  Text Text Text Text Text Text  Text Text.  \citealp{Boffelli03} might want to know about  text text text text
Text Text Text Text Text Text  Text Text Text Text Text Text Text Text Text  Text Text Text Text Text Text. Figure \ref{fig:02} shows that the above method  Text Text Text Text  Text Text Text Text Text Text  Text Text.  \citealp{Boffelli03} might want to know about  text text text text
Text Text Text Text Text Text  Text Text Text Text Text Text Text Text Text  Text Text Text Text Text Text. Figure \ref{fig:02} shows that the above method  Text Text Text Text  Text Text Text Text Text Text  Text Text.  \citealp{Boffelli03} might want to know about  text text text text


Text Text Text Text Text Text  Text Text Text Text Text Text Text Text Text  Text Text Text Text Text Text. Figure \ref{fig:02} shows that the above method  Text Text Text Text  Text Text Text Text Text Text  Text Text.  \citealp{Boffelli03} might want to know about  text text text text
Text Text Text Text Text Text  Text Text Text Text Text Text Text Text Text  Text Text Text Text Text Text. Figure \ref{fig:02} shows that the above method  Text Text Text Text  Text Text Text Text Text Text  Text Text.  \citealp{Boffelli03} might want to know about  text text text text
Text Text Text Text Text Text  Text Text Text Text Text Text Text Text Text  Text Text Text Text Text Text. Figure \ref{fig:02} shows that the above method  Text Text Text Text  Text Text Text Text Text Text  Text Text.  \citealp{Boffelli03} might want to know about  text text text text



\begin{table}[!t]
\processtable{This is table caption\label{Tab:01}}
{\begin{tabular}{llll}\toprule
head1 & head2 & head3 & head4\\\midrule
row1 & row1 & row1 & row1\\
row2 & row2 & row2 & row2\\
row3 & row3 & row3 & row3\\
row4 & row4 & row4 & row4\\\botrule
\end{tabular}}{This is a footnote}
\end{table}

\end{methods}

\begin{figure}[!tpb]%figure1
%\centerline{\includegraphics{fig01.eps}}
\caption{Caption, caption.}\label{fig:01}
\end{figure}

\begin{figure}[!tpb]%figure2
%\centerline{\includegraphics{fig02.eps}}
\caption{Caption, caption.}\label{fig:02}
\end{figure}

\section{Discussion}

Text Text Text Text Text Text  Text Text Text Text Text Text Text Text Text  Text Text Text Text Text Text. Figure \ref{fig:02} shows that the above method  Text Text Text Text  Text Text Text Text Text Text  Text Text.  \citealp{Boffelli03} might want to know about  text text text text
Text Text Text Text Text Text  Text Text Text Text Text Text Text Text Text  Text Text Text Text Text Text. Figure \ref{fig:02} shows that the above method  Text Text Text Text  Text Text Text Text Text Text  Text Text.  \citealp{Boffelli03} might want to know about  text text text text
Text Text Text Text Text Text  Text Text Text Text.




Table~\ref{Tab:01} shows that Text Text Text Text Text  Text Text Text Text Text Text. Figure \ref{fig:02} shows that
the above method Text Text. Text Text Text  Text Text Text Text Text Text. Figure \ref{fig:02} shows that
the above method Text Text. Text Text Text  Text Text Text Text Text Text. Figure \ref{fig:02} shows that
the above method Text Text.









%%%%%%%%%%%%%%%%%%%%%%%%%%%%%%%%%%%%%%%%%%%%%%%%%%%%%%%%%%%%%%%%%%%%%%%%%%%%%%%%%%%%%
%
%     please remove the " % " symbol from \centerline{\includegraphics{fig01.eps}}
%     as it may ignore the figures.
%
%%%%%%%%%%%%%%%%%%%%%%%%%%%%%%%%%%%%%%%%%%%%%%%%%%%%%%%%%%%%%%%%%%%%%%%%%%%%%%%%%%%%%%






\section{Conclusion}

(Table~\ref{Tab:01}) Text Text Text Text Text Text  Text Text Text Text Text Text Text Text Text  Text Text Text Text Text Text. Figure \ref{fig:02} shows that the above method  Text Text Text Text  Text Text Text Text Text Text  Text Text.  \citealp{Boffelli03} might want to know about  text text text text
Text Text Text Text Text Text  Text Text Text Text Text Text Text Text Text  Text Text Text Text Text Text. Figure \ref{fig:02} shows that the above method  Text Text Text Text  Text Text Text Text Text Text  Text Text.  \citealp{Boffelli03} might want to know about  text text text text
Text Text Text Text Text Text  Text Text Text Text Text Text Text Text Text  Text Text Text Text Text Text. Figure \ref{fig:02} shows that the above method  Text Text Text Text  Text Text Text Text Text Text  Text Text.



Text Text Text Text Text Text  Text Text Text Text Text Text Text Text Text  Text Text Text Text Text Text. Figure \ref{fig:02} shows that the above method  Text Text Text Text  Text Text Text Text Text Text  Text Text.  \citealp{Boffelli03} might want to know about  text text text text





\begin{enumerate}
\item this is item, use enumerate
\item this is item, use enumerate
\item this is item, use enumerate
\end{enumerate}

Text Text Text Text Text Text  Text Text Text Text Text Text Text Text Text  Text Text Text Text Text Text. Figure \ref{fig:02} shows that the above method  Text Text Text Text  Text Text Text Text Text Text  Text Text.  \citealp{Boffelli03} might want to know about  text text text text
Text Text Text Text Text Text  Text Text Text Text Text Text Text Text Text  Text Text Text Text Text Text. Figure \ref{fig:02} shows that the above method  Text Text Text Text  Text Text Text Text Text Text  Text Text.  \citealp{Boffelli03} might want to know about  text text text text
Text Text Text Text Text Text  Text Text Text Text Text Text Text Text Text  Text Text Text Text Text Text.






Text Text Text Text Text Text  Text Text Text Text Text Text Text Text Text  Text Text Text Text Text Text. Figure \ref{fig:02} shows that the above method  Text Text Text Text


\section*{Acknowledgement}
Text Text Text Text Text Text  Text Text.  \citealp{Boffelli03} might want to know about  text text text text

\paragraph{Funding\textcolon} Text Text Text Text Text Text  Text Text.

%\bibliographystyle{natbib}
%\bibliographystyle{achemnat}
%\bibliographystyle{plainnat}
%\bibliographystyle{abbrv}
%\bibliographystyle{bioinformatics}
%
%\bibliographystyle{plain}
%
%\bibliography{Document}


\begin{thebibliography}{}
\bibitem[Bofelli {\it et~al}., 2000]{Boffelli03} Bofelli,F., Name2, Name3 (2003) Article title, {\it Journal Name}, {\bf 199}, 133-154.

\bibitem[Bag {\it et~al}., 2001]{Bag01} Bag,M., Name2, Name3 (2001) Article title, {\it Journal Name}, {\bf 99}, 33-54.

\bibitem[Yoo \textit{et~al}., 2003]{Yoo03}
Yoo,M.S. \textit{et~al}. (2003) Oxidative stress regulated genes
in nigral dopaminergic neurnol cell: correlation with the known
pathology in Parkinson's disease. \textit{Brain Res. Mol. Brain
Res.}, \textbf{110}(Suppl. 1), 76--84.

\bibitem[Lehmann, 1986]{Leh86}
Lehmann,E.L. (1986) Chapter title. \textit{Book Title}. Vol.~1, 2nd edn. Springer-Verlag, New York.

\bibitem[Crenshaw and Jones, 2003]{Cre03}
Crenshaw, B.,III, and Jones, W.B.,Jr (2003) The future of clinical
cancer management: one tumor, one chip. \textit{Bioinformatics},
doi:10.1093/bioinformatics/btn000.

\bibitem[Auhtor \textit{et~al}. (2000)]{Aut00}
Auhtor,A.B. \textit{et~al}. (2000) Chapter title. In Smith, A.C.
(ed.), \textit{Book Title}, 2nd edn. Publisher, Location, Vol. 1, pp.
???--???.

\bibitem[Bardet, 1920]{Bar20}
Bardet, G. (1920) Sur un syndrome d'obesite infantile avec
polydactylie et retinite pigmentaire (contribution a l'etude des
formes cliniques de l'obesite hypophysaire). PhD Thesis, name of
institution, Paris, France.

\end{thebibliography}
\end{document}
