
\documentclass{bioinfo}
\copyrightyear{2015}
\pubyear{2015}

\usepackage{listings}
\usepackage{color}
\definecolor{lightgray}{rgb}{.9,.9,.9}
\definecolor{darkgray}{rgb}{.4,.4,.4}
\definecolor{purple}{rgb}{0.65, 0.12, 0.82}
\lstdefinelanguage{JavaScript}{
  keywords={break, case, catch, continue, debugger, default, delete, do, else, false, finally, for, function, if, in, instanceof, new, null, return, switch, this, throw, true, try, typeof, var, void, while, with},
  morecomment=[l]{//},
  morecomment=[s]{/*}{*/},
  morestring=[b]',
  morestring=[b]",
  ndkeywords={class, export, boolean, throw, implements, import, this},
  keywordstyle=\color{blue}\bfseries,
  ndkeywordstyle=\color{darkgray}\bfseries,
  identifierstyle=\color{black},
  commentstyle=\color{purple}\ttfamily,
  stringstyle=\color{red}\ttfamily,
  sensitive=true
}

\lstset{
   language=JavaScript,
   backgroundcolor=\color{lightgray},
   extendedchars=true,
   basicstyle=\footnotesize\ttfamily,
   showstringspaces=false,
   showspaces=false,
   numbers=left,
   numberstyle=\footnotesize,
   numbersep=9pt,
   tabsize=2,
   breaklines=true,
   showtabs=false,
   captionpos=b
}
\usepackage[ngerman]{babel}
%\usepackage[utf8]{inputenc}
\begin{document}
\raggedright
\firstpage{1}

\title[short Title]{Javascript Technology: Module Pattern}
\author[Sample \textit{et~al}]{Corresponding Author\,$^{1,*}$, Co-Author\,$^{2}$ and Co-Author\,$^2$\footnote{to whom correspondence should be addressed}}
\address{$^{1}$Department of XXXXXXX, Address XXXX etc.\\
$^{2}$Department of XXXXXXXX, Address XXXX etc.}

\history{Received on XXXXX; revised on XXXXX; accepted on XXXXX}

\editor{Associate Editor: XXXXXXX}

\maketitle

\begin{abstract}

\section{Motivation:}
This paper provides an in depth description of the Javascript Module Pattern. The Module Pattern is a way to implement classlike behaviour in Javascript by using the closure of anonymous functions to provide privacy and scope. It also explains the advantages and disadvantages that come with using the Module Pattern for your application. Finally it presents an application that has used the Module Pattern.

\section{Contact:} \href{name@bio.com}{name@bio.com}
\end{abstract}

\section{Introduction}
Logically separating pieces of code allows for easier reading and understanding of said code. It is common for programming languages to implement this through object orientation.
\vspace{\baselineskip}
Since Javascript was not designed with object orientation in mind certain hacks are required. The Module Pattern is one such hack. It implements classlike behaviour by allowing public and private properties in one single datastructure. This keeps the properties out of the global namespace and prevents unwanted modification from outside the structure. The Module Pattern is an entirely separate alternative to Javascript prototype based class emulation.


\section{Module Pattern:}


\subsection{Overview:}
As mentioned in the Introduction, the Module Pattern allows public and private properties in one datastructure. This is achieved by creating an object inside an anonymous function, called "module" in the code example, that is executed immediately after it's definition. The object is returned at the end of the function. Everything in this function exists inside a closure, providing privacy and state.  \\

\begin{flushleft}
All public methods and variables are defined as part of the object, all private ones are created independent of it. The return value is saved in a variable thus allowing it's properties to be accessed from outside the functions scope. The state provided by the closure remains consistent over all invocations of the Module. The following codes illustrates the basic Module Pattern.
\end{flushleft}

\vspace{\baselineskip}
\vspace{\baselineskip}
\vspace{\baselineskip}

\begin{lstlisting}
//define anonymous function
var My_Module = (function(){
	var module = {};
	var private_variable = 3;
	module.public_variable = 9;

	//return public part of the module
	return module;

//execute anonymous function immediately
]());
\end{lstlisting}

Public properties can now be accessed like this:

\begin{lstlisting}
My_Module.public_property
\end{lstlisting}


While private properties can not be accessed from outside the anonymous functions closure, the following code will not produce an error, but instead create a new public variable named "private\_variable"
\begin{lstlisting}
My_Module.private_variable
\end{lstlisting}

\subsection{Strengths and Weaknesses of the Module Pattern:}
A big advantage of the Module Pattern is scalabilty. Modules are isolated pieces of code and can be added or removed fairly easily since they are mostly independent of other code. The isolation also allows for a simple distribution of work among several programmers as they can be assigned different Modules to implement and can work separately.\\

\begin{flushleft}
Restricting variables to a local scope leads to less clutter in the global namespace. This, in addition to the fact that the public variables are bound to one module variable, prevents variable name conflicts which can be a problem when importing libraries or working with a team of developers. On top of that Modules can be augmented to add more methods and variables when required.
\end{flushleft}

The Module Pattern also has a few downsides. For one, inheritance requires the inheriting Module to explicitly copy all properties of the super Module. Also it is not possible to manipulate the private properties of a Module from outside the Module's scope. Not even while augmenting it. Since private properties only exist within the anonymous function's closure they can not be accessed from the outside at all. \vspace{\baselineskip}

Another problem is that changing the visibility of a property requires the programmer to edit every line of code that contains said property. This is the case because visibility is not defined by a single keyword but by whether the property is part of the returned object inside the Module so it is either accessed by 'module.property' if it is public or just 'property' if it is private.

%add code examples for strengths/weaknesses

\subsection{Global Variables:}
Global variables can make code hard to read or maintain since it is difficult for humans to determine where in the code they are used. The Module Pattern allows to import global variables into a Module explicitly by using them as arguments for the Module's anonymous function. Of course global variables can be accessed inside the functions closure regardless of whether they are imported explicitly or not. Using global variables as arguments is done for increased readability and is highly recommended\cite{adequatelygood}\\

\begin{lstlisting}
//use global variable as function parameter
var My_Module = (function(global_variable){
	var module = {};

	//access global variable
	module.global_incremented = function(){
		return global_variable + 1;
	}

	return module;

//use global variable as argument
](global_variable));
\end{lstlisting}

\subsection{Inheritance:}

Inheritance allows a new Module to derive all public properties from another Module. It is not only useful to reduce the lines of code in a program but it also increases readablity as related Modules are clearly recognizable as such.\vspace{\baselineskip}


Modules inherit from other Modules by copying all their non-private properties. To achieve this the super Module's public part is imported into the inheriting Module as an argument. A reference to the super Module is saved and all it's properties are recreated.\vspace{\baselineskip}

\begin{lstlisting}
//define anonymous function with
//super module as parameter
var My_Module = (function(super_module){
	var module = {};

	//create reference to super module
	module.super = super_module;

	//copy all properties from the
	//super module to the new module
	for (key in super_module) {
		if (super_module.hasOwnProperty(key) {
			module[key] = super_module[key];
		}
	}

//import super module as argument
//to new anonymous function
](super_module));
\end{lstlisting}


\subsection{Augmentation:}
As was briefly touched upon in the "Strengths and Weaknesses of the Module Pattern" section, Modules can be augmented after their original definition. A new anonymous function is defined and executed immediately. It takes the variable that held the original Module as an argument. Instead of creating a new object to contain all public properties, new public properties are added to the original modules object. The public properties of the original Module can also be modified. The original modules object is returned at the end of the function and the return value is used to overwrite the variable that held the original Module's public properties. The new public properties can now be accessed along the original ones through this variable. \\


\begin{lstlisting}
//define anonymous function with
//original Module as parameter
var Original_Module = (function(original_Module){

	//add new properties
	original_Module.new_variable = 25;
	original_Module.new_method = function(){
		return 3+5;
	};

	//overwrite original properties
	original_Module.original_variable = 18;
	original_Module.original_method = function(){
		return 4+2;
	}

	return original_Module;

//import original Module as argument
//to new anonymous function
](Original_Module));
\end{lstlisting}



It is worth noting that the original private properties can not be accessed in the new anonymous function, since they are only accessible from inside the original anonymous functions closure. New anonymous properties can be defined but they only exist in their own functions closure and can not be accessed from anywhere else.\vspace{\baselineskip}

These augmentations can be even be done from different files and can even be used when the original Module has not been created. This will be explained in detail in the following section.







\subsection{Advanced Module Pattern concepts}
Here we would like to introduce you to some advanced Module Pattern concepts like separating your module on to different files.
Using more files for one module can open you a whole bunch of new possibilities, for example it is simplifying the project structure.
We will also talk about the loading order of the files and the possibilities you have there.
In the end we'll show you how to manage the private scope over several files.

\subsubsection{One Module Pattern in different files}
Like we mentioned above its possible to organize your modules different tasks in more files.
But more important is that its possible now to independently load your module
extensions. That means if you can't discover which module extensions you need, at page load,
you can do it now dynamically as soon as you know it. Its only necessary to
include a dynamic loading framework and you can use it. This means you can load your different
strategies dynamically and on user needs. Reducing data traffic is one good reason to use this
concept.

\subsubsection{Augmentation possibilities}
By the use of the splitting to different files concept you have to choose between two different
augmentation possibilities:
\begin{itemize}
    \item loose augmentation (no defined loading order)
    \item tight augmentation (a defined loading order by your code)
\end{itemize}
This augmentation possibilities differ in the function argument.
The loose augmentation has the following structure:

% var module_name = function(module) {
% ...
% return module
% } (module_name || {})

This means if the module is already instanced we use it to extend and overwrite it.
Otherwise we use a new empty object. If you use loose augmentation the var keyword
is mandatory. The loose augmentation has no strict loading order and its possible to load them
as you wish.

On the other side there is the tight augmentation structured like this:

% [var] module_name = function(module) {
% ...
% return module
% } (module_name)

Here the var keyword is optional and not needed. The base of your module isn't structured
like this but has the base structure without the module as an argument. Only the extensions for
that base module need this tight augmentation structure. This implies that the base of
your module has to be loaded before the extensions. But for the extensions there is no defined
loading order.

There is a problem you could face if you load the extensions randomly, because of some relations
betweeen your extensions. Thats a really important thing you have to consider. So if there are
constraints related to your module extensions, its necessary to adapt also your loading
order in order to match the constraints.

\subsubsection{Cross-File private state}
So now you probably want to share your private variables, which normally are only accessible
inside the generator function, to other extensions of your module. Even if you have
more module extensions in one file (in my opinion there is no point in that, but netherless),
it isn't possible to access private variables of other extensions.

But there is a solution for this problem: you create a private and public object (here named \_private), both point
to the same object. Its properties are in the cross-file private scope. After the initialization
you have to call a ``seal'' function (called \_seal in our example) which deletes the public
variables \_private, \_seal and \_unseal. The unseal function (called \_unseal in our example)
recreates the public variables \_private, \_seal and \_unseal. To fullfill all this
requirements the following code has to be in every module extension function.

% Code für cross-file private state

This code creates creates a local ``private'' object which points to the same as the public ``private''
object. They both point to the public ``private'' object if it exists or to a new object.
The same workflow is valid for the ``seal'' and the ``unseal'' functions, if the public function
exists point to it, if not create the function.

Important is also if its necessary to load an extension not directly on page loading, but
later on, before loading it the ``unseal'' function must be called and afterwards the
``seal'' function. Now you have a full working cross-file private scope.

\section{Our Web App - Tic Tac Toe Overview:}
We developed a small web app where its possible to play the famous Tic Tac Toe game.
For a better layout and to simplify our lifes we added the Bootstrap framework. It is
a Open source HTML, CSS and JavaScript framework and they advertise it as the most popular.
To load our files dynamically we used the 3rd party framework LAB.js.

\section{How Module Pattern Helped Our App (better title pls, jeebus)}

This is better looking sample text compared to the original sample text that just repeated the word "Text". This is better looking sample text compared to the original sample text that just repeated the word "Text". This is better looking sample text compared to the original sample text that just repeated the word "Text".


\section{Conclusion/Summary}

This is better looking sample text compared to the original sample text that just repeated the word "Text". This is better looking sample text compared to the original sample text that just repeated the word "Text". This is better looking sample text compared to the original sample text that just repeated the word "Text".
%Einleuitung / Was ist ModulePattern / Was ist daran gut, was schlecht / vll paar Beispiele wie man das elegant anwendet / beschreibung unserer WebApp und warum das pattern da geholfen hat / Schluss

\section{everything past this line is a relict from the default template}

\section{Approach}

Equation~(\ref{eq:01}) Text Text Text Text Text Text  Text Text Text Text Text Text Text Text Text  Text Text Text Text Text Text. Figure \ref{fig:02} shows that the above method  Text Text Text Text  Text Text Text Text Text Text  Text Text.  \citealp{Boffelli03} might want to know about  text text text text ��


\begin{methods}
\section{Methods}

Text Text Text Text Text Text  Text Text Text Text Text Text Text Text Text  Text Text Text Text Text Text. Figure \ref{fig:02} shows that the above method  Text Text Text Text  Text Text Text Text Text Text  Text Text.  \citealp{Boffelli03} might want to know about  text text text text
Text Text Text Text Text Text  Text Text Text Text Text Text Text Text Text  Text Text Text Text Text Text. Figure \ref{fig:02} shows that the above method  Text Text Text Text  Text Text Text Text Text Text  Text Text.  \citealp{Boffelli03} might want to know about  text text text text
Text Text Text Text Text Text  Text Text Text Text Text Text Text Text Text  Text Text Text Text Text Text. Figure \ref{fig:02} shows that the above method  Text Text Text Text  Text Text Text Text Text Text  Text Text.  \citealp{Boffelli03} might want to know about  text text text text

\begin{itemize}
\item for bulleted list, use itemize
\item for bulleted list, use itemize
\item for bulleted list, use itemize
\end{itemize}



Text Text Text Text Text Text  Text Text Text Text Text Text Text Text Text  Text Text Text Text Text Text. Figure \ref{fig:02} shows that the above method  Text Text Text Text  Text Text Text Text Text Text  Text Text.  \citealp{Boffelli03} might want to know about  text text text text
Text Text Text Text Text Text  Text Text Text Text Text Text Text Text Text  Text Text Text Text Text Text. Figure \ref{fig:02} shows that the above method  Text Text Text Text  Text Text Text Text Text Text  Text Text.  \citealp{Boffelli03} might want to know about  text text text text
Text Text Text Text Text Text  Text Text Text Text Text Text Text Text Text  Text Text Text Text Text Text. Figure \ref{fig:02} shows that the above method  Text Text Text Text  Text Text Text Text Text Text  Text Text.  \citealp{Boffelli03} might want to know about  text text text text
Text Text Text Text Text Text  Text Text Text Text Text Text Text Text Text  Text Text Text Text Text Text. Figure \ref{fig:02} shows that the above method  Text Text Text Text  Text Text Text Text Text Text  Text Text.  \citealp{Boffelli03} might want to know about  text text text text
Text Text Text Text Text Text  Text Text Text Text Text Text Text Text Text  Text Text Text Text Text Text.


Text Text Text Text Text Text  Text Text Text Text Text Text Text Text Text  Text Text Text Text Text Text. Figure \ref{fig:02} shows that the above method  Text Text Text Text  Text Text Text Text Text Text  Text Text.  \citealp{Boffelli03} might want to know about  text text text text
Text Text Text Text Text Text  Text Text Text Text Text Text Text Text Text  Text Text Text Text Text Text. Figure \ref{fig:02} shows that the above method  Text Text Text Text  Text Text Text Text Text Text  Text Text.  \citealp{Boffelli03} might want to know about  text text text text
Text Text Text Text Text Text  Text Text Text Text Text Text Text Text Text  Text Text Text Text Text Text. Figure \ref{fig:02} shows that the above method  Text Text Text Text  Text Text Text Text Text Text  Text Text.  \citealp{Boffelli03} might want to know about  text text text text



Text Text Text Text Text Text  Text Text Text Text Text Text Text Text Text  Text Text Text Text Text Text. Figure \ref{fig:02} shows that the above method  Text Text Text Text  Text Text Text Text Text Text  Text Text.  \citealp{Boffelli03} might want to know about  text text text text
Text Text Text Text Text Text  Text Text Text Text Text Text Text Text Text  Text Text Text Text Text Text. Figure \ref{fig:02} shows that the above method  Text Text Text Text  Text Text Text Text Text Text  Text Text.  \citealp{Boffelli03} might want to know about  text text text text
Text Text Text Text Text Text  Text Text Text Text Text Text Text Text Text  Text Text Text Text Text Text. Figure \ref{fig:02} shows that the above method  Text Text Text Text  Text Text Text Text Text Text  Text Text.  \citealp{Boffelli03} might want to know about  text text text text


Text Text Text Text Text Text  Text Text Text Text Text Text Text Text Text  Text Text Text Text Text Text. Figure \ref{fig:02} shows that the above method  Text Text Text Text  Text Text Text Text Text Text  Text Text.  \citealp{Boffelli03} might want to know about  text text text text
Text Text Text Text Text Text  Text Text Text Text Text Text Text Text Text  Text Text Text Text Text Text. Figure \ref{fig:02} shows that the above method  Text Text Text Text  Text Text Text Text Text Text  Text Text.  \citealp{Boffelli03} might want to know about  text text text text
Text Text Text Text Text Text  Text Text Text Text Text Text Text Text Text  Text Text Text Text Text Text. Figure \ref{fig:02} shows that the above method  Text Text Text Text  Text Text Text Text Text Text  Text Text.  \citealp{Boffelli03} might want to know about  text text text text



\begin{table}[!t]
\processtable{This is table caption\label{Tab:01}}
{\begin{tabular}{llll}\toprule
head1 & head2 & head3 & head4\\\midrule
row1 & row1 & row1 & row1\\
row2 & row2 & row2 & row2\\
row3 & row3 & row3 & row3\\
row4 & row4 & row4 & row4\\\botrule
\end{tabular}}{This is a footnote}
\end{table}

\end{methods}

\begin{figure}[!tpb]%figure1
%\centerline{\includegraphics{fig01.eps}}
\caption{Caption, caption.}\label{fig:01}
\end{figure}

\begin{figure}[!tpb]%figure2
%\centerline{\includegraphics{fig02.eps}}
\caption{Caption, caption.}\label{fig:02}
\end{figure}

\section{Discussion}

Text Text Text Text Text Text  Text Text Text Text Text Text Text Text Text  Text Text Text Text Text Text. Figure \ref{fig:02} shows that the above method  Text Text Text Text  Text Text Text Text Text Text  Text Text.  \citealp{Boffelli03} might want to know about  text text text text
Text Text Text Text Text Text  Text Text Text Text Text Text Text Text Text  Text Text Text Text Text Text. Figure \ref{fig:02} shows that the above method  Text Text Text Text  Text Text Text Text Text Text  Text Text.  \citealp{Boffelli03} might want to know about  text text text text
Text Text Text Text Text Text  Text Text Text Text.




Table~\ref{Tab:01} shows that Text Text Text Text Text  Text Text Text Text Text Text. Figure \ref{fig:02} shows that
the above method Text Text. Text Text Text  Text Text Text Text Text Text. Figure \ref{fig:02} shows that
the above method Text Text. Text Text Text  Text Text Text Text Text Text. Figure \ref{fig:02} shows that
the above method Text Text.









%%%%%%%%%%%%%%%%%%%%%%%%%%%%%%%%%%%%%%%%%%%%%%%%%%%%%%%%%%%%%%%%%%%%%%%%%%%%%%%%%%%%%
%
%     please remove the " % " symbol from \centerline{\includegraphics{fig01.eps}}
%     as it may ignore the figures.
%
%%%%%%%%%%%%%%%%%%%%%%%%%%%%%%%%%%%%%%%%%%%%%%%%%%%%%%%%%%%%%%%%%%%%%%%%%%%%%%%%%%%%%%






\section{Conclusion}

(Table~\ref{Tab:01}) Text Text Text Text Text Text  Text Text Text Text Text Text Text Text Text  Text Text Text Text Text Text. Figure \ref{fig:02} shows that the above method  Text Text Text Text  Text Text Text Text Text Text  Text Text.  \citealp{Boffelli03} might want to know about  text text text text
Text Text Text Text Text Text  Text Text Text Text Text Text Text Text Text  Text Text Text Text Text Text. Figure \ref{fig:02} shows that the above method  Text Text Text Text  Text Text Text Text Text Text  Text Text.  \citealp{Boffelli03} might want to know about  text text text text
Text Text Text Text Text Text  Text Text Text Text Text Text Text Text Text  Text Text Text Text Text Text. Figure \ref{fig:02} shows that the above method  Text Text Text Text  Text Text Text Text Text Text  Text Text.



Text Text Text Text Text Text  Text Text Text Text Text Text Text Text Text  Text Text Text Text Text Text. Figure \ref{fig:02} shows that the above method  Text Text Text Text  Text Text Text Text Text Text  Text Text.  \citealp{Boffelli03} might want to know about  text text text text





\begin{enumerate}
\item this is item, use enumerate
\item this is item, use enumerate
\item this is item, use enumerate
\end{enumerate}

Text Text Text Text Text Text  Text Text Text Text Text Text Text Text Text  Text Text Text Text Text Text. Figure \ref{fig:02} shows that the above method  Text Text Text Text  Text Text Text Text Text Text  Text Text.  \citealp{Boffelli03} might want to know about  text text text text
Text Text Text Text Text Text  Text Text Text Text Text Text Text Text Text  Text Text Text Text Text Text. Figure \ref{fig:02} shows that the above method  Text Text Text Text  Text Text Text Text Text Text  Text Text.  \citealp{Boffelli03} might want to know about  text text text text
Text Text Text Text Text Text  Text Text Text Text Text Text Text Text Text  Text Text Text Text Text Text.






Text Text Text Text Text Text  Text Text Text Text Text Text Text Text Text  Text Text Text Text Text Text. Figure \ref{fig:02} shows that the above method  Text Text Text Text


\section*{Acknowledgement}
Text Text Text Text Text Text  Text Text.  \citealp{Boffelli03} might want to know about  text text text text

\paragraph{Funding\textcolon} Text Text Text Text Text Text  Text Text.

%\bibliographystyle{natbib}
%\bibliographystyle{achemnat}
%\bibliographystyle{plainnat}
%\bibliographystyle{abbrv}
%\bibliographystyle{bioinformatics}
%
%\bibliographystyle{plain}
%
%\bibliography{Document}


\begin{thebibliography}{}


\bibitem{Javascript Design Patterns} Addy Osmani: \emph{Learning JavaScript Design Patterns}, O'Reilly Media \verb|http://addyosmani.com/resources/essentialjsdesignpatterns/book/#modulepatternjavascript|

\bibitem{adequatelygood} \emph{adequatelygood.com on the Module Pattern, May 8th '15}, \verb|http://www.adequatelygood.com/JavaScript-Module-Pattern-In-Depth.html|


\bibitem[Bofelli {\it et~al}., 2000]{Boffelli03} Bofelli,F., Name2, Name3 (2003) Article title, {\it Journal Name}, {\bf 199}, 133-154.

\bibitem[Bag {\it et~al}., 2001]{Bag01} Bag,M., Name2, Name3 (2001) Article title, {\it Journal Name}, {\bf 99}, 33-54.

\bibitem[Yoo \textit{et~al}., 2003]{Yoo03}
Yoo,M.S. \textit{et~al}. (2003) Oxidative stress regulated genes
in nigral dopaminergic neurnol cell: correlation with the known
pathology in Parkinson's disease. \textit{Brain Res. Mol. Brain
Res.}, \textbf{110}(Suppl. 1), 76--84.

\bibitem[Lehmann, 1986]{Leh86}
Lehmann,E.L. (1986) Chapter title. \textit{Book Title}. Vol.~1, 2nd edn. Springer-Verlag, New York.

\bibitem[Crenshaw and Jones, 2003]{Cre03}
Crenshaw, B.,III, and Jones, W.B.,Jr (2003) The future of clinical
cancer management: one tumor, one chip. \textit{Bioinformatics},
doi:10.1093/bioinformatics/btn000.

\bibitem[Auhtor \textit{et~al}. (2000)]{Aut00}
Auhtor,A.B. \textit{et~al}. (2000) Chapter title. In Smith, A.C.
(ed.), \textit{Book Title}, 2nd edn. Publisher, Location, Vol. 1, pp.
???--???.

\bibitem[Bardet, 1920]{Bar20}
Bardet, G. (1920) Sur un syndrome d'obesite infantile avec
polydactylie et retinite pigmentaire (contribution a l'etude des
formes cliniques de l'obesite hypophysaire). PhD Thesis, name of
institution, Paris, France.

\end{thebibliography}
\end{document}
